\documentclass{proposalnsf}
% This class file has been tweaked to death by LBarba to fit precisely the 
% formatting strictures of NSF, while still being rather pretty.


%--------------------------------------------------------------------  PROCESS WITH XeLaTeX
\usepackage{fontspec}% provides font selecting commands 
\usepackage{paralist}       % compactitem environment
\usepackage{xunicode}% provides unicode character macros 
\usepackage{xltxtra} % provides some fixes/extras 
\setromanfont[Mapping=tex-text,
                 SmallCapsFont={Palatino},
                 SmallCapsFeatures={Scale=0.85}]{Palatino}
\setsansfont[Scale=0.85]{Trebuchet MS} 
\setmonofont[Scale=0.85]{Monaco}

\renewcommand{\captionlabelfont}{\bf\sffamily}
\usepackage[hang,flushmargin]{footmisc} 
% 'hang' flushes the footnote marker to the left,  'flushmargin'  flushes the text as well.



% Define the color to use in links:
\definecolor{linkcol}{rgb}{0.459,0.071,0.294}
\definecolor{sectcol}{rgb}{0.63,0.16,0.16} % {0,0,0}
\definecolor{propcol}{rgb}{0.75,0.0,0.04}

\definecolor{gray}{rgb}{0.25,0.25,0.25}
\definecolor{ngreen}{rgb}{0.7,0.7,0.7} % a darker shade of green



\usepackage[
    xetex,
    pdftitle={NSF proposal},
    pdfauthor={Rachel Slaybaugh, Kaitlin Thaney, Lorena Barba, C. Titus Brown, Paul Wilson, Ethan White, Tracy Teal, and Greg Wilson},
    pdfpagemode={UseOutlines},
    pdfpagelayout={TwoColumnRight},
    bookmarks, bookmarksopen,bookmarksnumbered={True},
    pdfstartview={FitH},
    colorlinks, linkcolor={sectcol},citecolor={sectcol},urlcolor={sectcol}
    ]{hyperref}

%% Define a new style for the url package that will use a smaller font.
\makeatletter
\def\url@leostyle{%
  \@ifundefined{selectfont}{\def\UrlFont{\sf}}{\def\UrlFont{\small\ttfamily}}}
\makeatother
%% Now actually use the newly defined style.
\urlstyle{leo}


% this handles hanging indents for publications
\def\rrr#1\\{\par
\medskip\hbox{\vbox{\parindent=2em\hsize=6.12in
\hangindent=4em\hangafter=1#1}}}


\addto\captionsamerican{%
  \renewcommand{\refname}%
    {References Cited}%
} % solution found here: http://www.tex.ac.uk/cgi-bin/texfaq2html?label=latexwords

\def\baselinestretch{1}
\setlength{\parindent}{0mm} \setlength{\parskip}{0.8em}

\newlength{\up}
\setlength{\up}{-4mm}

\newlength{\hup}
\setlength{\hup}{-2mm}

\sectionfont{\large\bfseries\color{sectcol}\vspace{-2mm}}
\subsectionfont{\normalsize\it\bfseries\vspace{-4mm}}
\subsubsectionfont{\normalsize\mdseries\itshape\vspace{-4mm}} %\itshape
\paragraphfont{\bfseries}
% ---------------------------------------------------------------------
% DRAFTING COMMENTS:
\newcommand\ignore[1]{} % Styles for author comments:
% Enter a comment like this:   \comment{This is a comment.}
%\ignore{
\newcommand{\important}[1]{\textcolor{red}{ #1 }}
\newcommand{\comment}[1]{\textcolor{blue}{ #1 }}
%}
% Uncomment lines below to change from visible to invisible comments.
%\renewcommand{\important}[1]{}
%\renewcommand{\comment}[1]{}

\begin{document}

% ----------------------  PROJECT SUMMARY - 1 page max - write in 3rd person


% Small title with normal-size font:
\begin{center}

\comment{LB: please change title to reflect fact that there's some
  design in this as well as assessment (reference the fact that we're
  hiring a half-time instructional designer)}

\comment{GW: highlight the fact that we're hiring a full-time
  educational assessment expert at UCB: wasn't clear to LB; make it
  clear that this is Design and Development Research (pg 12) and
  Effectiveness Research (pg 14), see
  http://www.nsf.gov/pubs/2013/nsf13126/nsf13126.pdf}

\comment{ME: I think you need to go into much more detail on the
  workshops to the point of outlining a schedule for the 2 days with
  the caveat that each workshop will be tailored to the needs of the
  population. You also want to mention who will be giving the
  workshops-if there are multiple speakers, state that. I wrote an REU
  proposal for with the ME dept and one of the activities is a 2 day
  Lab Fundamentals workshop and we created a schedule and described
  each session (ie how to conduct a literature review) and who will be
  running these sessions (grad students to be hired - you don't have
  to name names, but be specific about who the responsiblity will fall
  to.}

\comment{ME: The workshops will be given to REU students at each insitution
    correct? If that is the case, identify the REU's on each campus
    and mention if there are any currently under review at
    NSF. Support from the REU director/PI would also help since the
    proposal is relying on providing this workshop to this
    population. You want a commitment that your workshop will be
    incorporated into their summer schedule. At Berkeley, we have a
    number of REU's, primarily in the EECS dept. Once I read the
    proposal in more detail, I can let you know which of our programs
    might be a good fit.}

\comment{ME: Since this is a proposal among multiple campuses, it would greatly
    strengthen the proposal if there is a way to bring everyone
    together, not necessarily in person as the expense is likely
    outside the budget of the proposal, but perhaps a a video
    conference after the summer workshops for students to exchange
    ideas/feedback or another virtual meeting that brings everyone
    together.}

\large\textbf{Assessing the Impact of Intensive Software Skills Training
  on Students' Scientific Careers}

{\large \sf Rachel Slaybaugh, Kaitlin Thaney, Lorena Barba, C. Titus Brown, Paul Wilson\\ 
  with Ethan White, Tracy Teal, and Greg Wilson}

\end{center}

\textbf{Overview:}

Computers are now ubiquitous in scientific research, but many
researchers in science, technology, engineering, and mathematics
(STEM) are still not \emph{computationally competent}: they do
repetitive tasks manually instead of automating them, develop software
haphazardly, and fail to track their work in any systematic,
reproducible way.  This lack of foundational skills impedes their
ability to do research, and prevents them from engaging with new
opportunities in open and web-enabled science.

The aim of this project is to train STEM undergraduates in these
skills, and to assess the impact of that training on their
productivity and career paths.  We will do this by adapting the
workshops run by Software Carpentry for researchers at the graduate
level and above to undergraduates, and track their alumni over several
subsequent years both to improve the training itself, and to encourage
wider adoption of our model.

More specifically, we will run two-day software skills workshops each
year for five years for undergraduate students taking part in that
year in the NSF's Research Experience for Undergraduates (REU) program
and similar programs, at or near the start of those students' time in
the lab.  We will additionally run at least one workshop each year
specifically aimed at female students, and another specifically aimed
at students from minority groups that are under-represented in STEM.

These workshops, which will reach over 2000 students during the course
of the project, will help participants be more productive during their
REU, and also prepare them to work in a world where all aspects of
science are increasingly dependent on computing.  These undergraduates
will then serve as the treatment population for a five-year study of
the impact of this training on their careers in general, and their
involvement with open and web-enabled science in particular.

\textbf{Intellectual merit:}

Our aim is to assess the extent to which training of this kind can
accelerate the science done by recipients, and their careers.  We
expect to gain insight into whether students who receive this training
are more likely than their peers to continue to graduate school than
their peers, to incorporate open science and/or web-enabled science
tools and practices into their work, to choose computationally
oriented research topics and/or careers, and/or to develop new
computational tools and practices.  We also hope to determine the
extent to which training of this kind changes students' outlook on the
practice of science itself, and whether it can help level the playing
field for women and minorities currently underrepresented in computing
and science.

\textbf {Broader impacts:}

In the medium term, this project will have broader impact through the
creation and dissemination of teaching materials and practices: all of
the materials produced by and for this project will be made freely
available under the Creative Commons - Attribution (CC-BY) license,
while the results of our studies of the program's effect will be
shared with other educators through science education journals and
conferences.

In the long term, the project's greatest impact will be on scientific
competitiveness.  Computing is no longer optional in any part of
science: even scientists who don't think of themselves as doing
computational work rely on computers to prepare papers, store data,
and collaborate with colleagues.  The better their computing skills
are, the better able they will be to contribute to their research,
and to national economic competitiveness.

\renewcommand{\thepage} {\footnotesize Project Summary}

\newpage

% ----------------------  end SUMMARY 


% ----------------------  TECHNICAL PROPOSAL starts here ... Maximum 15 pages
\pagestyle{plain}
% reset page numbering to 1 
\pagenumbering{arabic}
\renewcommand{\thepage} {\arabic{page}}

\begin{center}
\small{PROJECT DESCRIPTION}

% Small title with normal-size font

\large\textbf{Assessing the Impact of Intensive Software Skills Training
  on Students' Scientific Careers}

{\large \sf Rachel Slaybaugh, Kaitlin Thaney, Lorena Barba, C. Titus Brown, Paul Wilson\\ 
  with Ethan White, Tracy Teal, and Greg Wilson}

\end{center}

\section{The Problem}

Scientists and engineers invented electronic computers to accelerate
their work, but two generations later, many researchers in science,
technology, engineering, and mathematics (STEM) are still not
\emph{computationally competent}: they do repetitive tasks manually
instead of automating them, develop software using a methodology best
summarized as ``copy, paste, tweak, and pray'', and fail to track
their work in any systematic, reproducible way.

While the World-Wide Web was created by a scientist to help his
peers share information, many still use it primarily as a way to find
and download PDFs.  Researchers may understand that open data can fuel
new insights, but often lack the skills needed to create and provide a
reusable data set.  Equally, any discussion of changing scientific
publishing, making research reproducible, or using the web to support
``science as a service'' must eventually address the lack of
pre-requisite skills in the general STEM research community.

Since the mid-1980s (at least), proponents of computational science
have taken an ``if we build it, they will come'' approach to this
problem.  We believe this learning-by-osmosis approach is unlikely to
bear fruit for several reasons:

\begin{enumerate}

\item
  \emph{The curriculum is full.}  Undergraduate STEM programs already
  struggle to cover material regarded as core to their field.  While
  many scientists would agree that more material on programming,
  reproducible research, or web-enabled science would be useful, there
  is no consensus on what to take out to make room.

\item
  \emph{The blind leading the blind.}  Many faculty lack computational
  skills themselves, and hence are unable to pass them on.

\item
  Scientists and software developers have different cultures,
  different priorities, and different approaches to problem solving,
  which often impedes collaboration and knowledge transfer
  \cite{segal2005a}.

\item
  \emph{Difficulty of assessing impact.} It is easy to say, ``This
  discovery could not have been made without use of that
  supercomputer.''  It is much harder to attribute specific advances
  in science to prior training in general computing skills.

\end{enumerate}

The final, and possibly largest, issue is that \emph{the rewards are
  unknown}.  Open, web-based science is still in its infancy, so there
is no general understanding of what people might need to know in order
to incorporate it into their research careers.  Since it is hard to
measure something if you don't know what to look for, or if it is so
young that there hasn't actually \emph{been} long-term impact, little
systematic study has been done to date of whether early training in
the skills needed to practice open, web-enabled science actually has
an impact, and if so, how and how much.  Without such feedback, there
is no systematic way to improve the training programs that currently
exist.

\section{Proposed Work}

\comment{GW: flesh out - LB thinks this is too thin now that stuff has
  been moved to the appendix.}

This proposal builds on the success to date of the Software Carpentry
workshops in enhancing the skills of graduate students, post-docs, and
faculty.  We propose to:

\begin{enumerate}

\item
  conduct formative evaluation of the impact of software skills
  training for undergraduates likely to continue in research careers
  as they progress through the early stages of those careers;

\item
  conduct summative evaluation of the training's overall impact on a
  multi-year timescale in order to improve the content and
  presentation of the training; and

\item
  disseminate the resulting curriculum widely.

\end{enumerate}

We will run software-skills workshops for undergraduate students
taking part each year in the NSF's Research Experience for
Undergraduates (REU) program and similar programs, at or near the
start of those students' time in the lab.  We expect this training
will help them be more productive during their REU (graduate-level
participants in our existing workshops typically report that what we
teach saves them a day a week), and also prepare them to work in a
world where all aspects of science are increasingly dependent on
computing.  These undergraduates will serve as the treatment
population for a five-year study of the impact of this training on
their careers in general, and their involvement with open and
web-enabled science in particular.

\subsection{Workshops}\vspace{\hup}

We will run two-day workshops at a steadily increasing number of sites
each year for four years, timed to coincide with the start of the
summer REU influx.  Each workshop will be offered to a minimum of 40
learners per site (giving us a target study population of 1440
students by year 4).  All of the workshop instructors will have been
trained and certified by Software Carpentry, and will have had prior
experience teaching this material.  The content will be tailored to
meet local needs, but will be based on what is being used at that time
by Software Carpentry and affiliated educational efforts.  By design,
it is straightforward to adapt workshop materials and contribute
changes back to the Software Carpentry organization.  These features
enhance the portability and flexibility of the workshops, and increase
the likelihood of wide dissemination beyond this project.

The home sites for investigators named in this proposal (Michigan
State University, Utah State University, George Washington University,
University of California -- Berkeley, and University of Wisconsin --
Madison) will run workshops in each of those years.  Other NSF REU
sites will be added each year, increasing the total to 15 in year 5,
which increases the size of our study population.

In order to increase the diversity of the study population, at least
one workshop in each year will be aimed specifically at female
students.  Software Carpentry's first such workshop, held in Boston in
June 2013, attracted 120 participants; its second is scheduled for
Lawrence Berkeley National Laboratory in March 2014, and at least two
more will be held by the time work on this project commences (one in
the United States and one in Europe).  This work will build on that
experience, and draw on the pool of instructors who have gained
mentoring experience through those specific workshops.

Finally, we will organize workshops in years 2--5 specifically aimed
at students from minority groups that are underrepresented in STEM.
We are already in contact with the Computing Alliance for
Hispanic-Serving Institutions (CAHSI), and with the Association of
Public and Land-grant Universities' program for historically black
colleges and universities (HBCUs); Software Carpentry is running its
first workshop at an HBCU (Spelman) in early 2014, and we expect to
have expanded these efforts by the start of this project.

\subsection{Assessment}\vspace{\hup}

We will employ one full-time researcher for five years to monitor and
compare undergraduate participants in these workshops, participants in
a subset of our regular (graduate-level) workshops, and
non-participants (as a control population).  Assessment will focus
particularly, but not exclusively, on the following questions:

\begin{enumerate}

\item
  Are students who receive this training more likely to continue to
  graduate school than their peers?

\item
  Are students who receive this training more likely than their peers
  to incorporate open science and/or web-enabled science tools and
  practices into their work?

\item
  Are students who receive this training more likely than their peers
  to choose computationally oriented research topics and/or careers?
  Are those who do not choose computationally oriented paths
  nevertheless more likely to incorporate the tools and practices
  mentioned above into their work?

\item
  Are students who receive this training more likely than their peers
  to develop new tools and practices, and/or become involved in
  outreach and education activities (i.e., are they more likely to
  become creators and leaders)?

\item
  Do outcomes differ between women and underrepresented minorities on
  one hand and non-underrepresented minorities and men on the other?
  If so, in what ways, and what steps are effective in correcting for
  these differences?

\item
  In what ways does this training change students' outlook on the
  practice of science itself?

\end{enumerate}

This researcher will also explore ways in which our engagement with
students changes the outlook and work practices of their peers and
faculty supervisors (i.e., whether there is knowledge transfer
sideways and upward), and at the effectiveness of the community
building and dissemination activities detailed in the next sections.

As with our work to date (discussed below), assessment will use both
qualitative and quantitative techniques.  On the qualitative side, we
will conduct a series of interviews over the five-year period of the
study to see how attitudes, aspirations, and activities change.
Quantitatively, we will measure uptake of key tools such as version
control as a proxy for adoption of related practices, as well as
exploring more traditional measures of research success such as
progression to graduate school and publication/citation rates.

Our findings will be shared with other researchers through publication
in peer-reviewed journals and high-profile conferences, as discussed
below.

\subsection{Community Building}\vspace{\hup}

We will employ one graduate student part-time at each participating
site each year to provide technical support to workshop participants,
and to act as an anchor for a Hacker Within-style grassroots group at
that site.  These community liaisons will not be study subjects, but
will help us stay in touch with students who are (a key requirement
for any longitudinal study).

Separately, the Mozilla Science Lab will focus part of its ongoing
community engagement efforts on the students who have taken part in
our workshops during both the remainder of their undergraduate careers
and afterward in order to ensure that they become part of the broader
open science community.  This may include helping the students
organize and run workshops of their own in subsequent years,
connecting them with other open science projects, introducing them to
potential graduate supervisors who understand and value their new
skills and outlook, etc.

As a subordinate part of their work, the researcher employed by this
project will assess the effectiveness of these local organizers.  In
particular, they will explore whether seeding activity in this way
leads to the formation of self-sustaining grassroots groups, and if
so, what activities those groups develop on their own, how (and how
effectively) they share discoveries with each other, the extent to
which alumni of this program stay engaged with these groups, and
whether the presence of these groups has a demonstrable impact on
students' career paths in general, and their engagement with open and
web-enabled science in particular.

\subsection{Curriculum Development and Dissemination}\vspace{\hup}

We will employ one instructional designer part-time during each of the
study's first four years to create new material, and to improve
existing material based on feedback from workshop participants and the
assessment program.  Here, ``creating material'' may include both
designing and implementing new domain-specific learning modules, and
translating existing materials into new forms, such as video
recordings of lectures or auto-graded quizzes for self-paced
instruction.  This work will be done in consultation with educators at
participating institutions in order to encourage incorporation of
those materials into existing curricula.

All of the materials produced by and for this project will be made
freely available under the Creative Commons -- Attribution (CC-BY)
license.  The instructional designer will work with the Mozilla
Science Lab and affiliated groups to share these materials, and the
results of our studies of the program's impact, through science
education journals, conferences, and other channels. Specifically, we
plan to publish articles in:

\begin{compactitem}

\item
  \emph{Science Education}

\item
  \emph{Physics Education}

\item
  \emph{Journal of Computers in Mathematics and Science Teaching}

\item
  \emph{American Scientist}

\end{compactitem}

and at the following conferences:

\begin{compactitem}

\item
  the Special Interest Group on Computer Science Education (SIGCSE) of the Association for Computing Machinery, Inc. (ACM)

\item
  the International Conference on Physics Education (ICPE)

\item
  the Association for Biology Laboratory Education (ABLE)

\item
  the Society for Industrial and Applied Mathematics (SIAM)'s education session at the Joint Mathematics Meeting

\item
  the Education Training and Workforce Development Division's track at the American Nuclear Society (ANS)'s annual meeting

\end{compactitem}

The dissemination of this project's curriculum has strong potential to
be high. Workshop materials are all open access and flexible, thus
they can be readily adopted by others. Adapting workshop materials is
low cost and does not require special equipment. Workshop materials
are structured such that they can scale to the size and application of
interest to a particular group. The Software Carpentry infrastructure
provides support in the form of materials and people. Anyone using
workshop materials can directly contribute changes and feedback, which
both increases buy-in and improves the materials organically. And
finally, local chapters of The Hacker Within create a natural
ecosystem of support for workshop participants, their peers, and
faculty.

As a subordinate part of their work, the researcher employed by this
project will assess the extent to which curriculum developed during
this program is taken up by other educators (particularly those who
think of themselves as scientists first and computationalists second),
and their perception of its utility.  Mid-point results of this
evaluation will be shared with the instructional designer in order to
allow evidence-based improvement of the materials.

\section{Related Work}

\subsection{Research}\vspace{\hup}

Studies of how scientists use computers and the web have found that
most scientists learn what they know about developing software and
using computers and the web in their research through osmosis and word
of mouth \cite{hannay2009,prabhu2011}. In our experience, most
training meant to address this issue:

\begin{itemize}

\item
  does not target scientists' specific needs (e.g., is a general
  ``Introduction to Computing'' class shared with students majoring in
  other areas);

\item
  only covers the mechanics of programming in a particular language
  rather than giving a complete picture including data management,
  web-enabled publishing, the ``defense in depth'' approach to
  correctness discussed in \cite{dubois2005}, or the other
  foundational skills laid out in \cite{wilson2013}; and/or

\item
  jumps to advanced topics such as parallel computing before
  scientists have mastered the foundations.  (Most research on
  scientific computing, such as \cite{hochstein2005}, does the same.)

\end{itemize}

\subsection{Software Carpentry}\vspace{\hup}

Software Carpentry \cite{swcsite,wilson2012} is the largest effort to
date to address these issues. Originally created as a training program
at Los Alamos National Laboratory in the late 1990s, it is now part of
the Mozilla Science Lab's efforts to help scientists take advantage of
ways in which the web can change the practice of science today, and
invent new ways tomorrow.  Over 100 certified volunteer instructors
delivered two-day intensive workshops to more than 4200 people in 2013
alone.  Their subject matter usually includes:

\begin{compactitem}
\item
  the Unix shell,
\item
  version control with Git,
\item
  programming in Python or R, and
\item
  using SQL.
\end{compactitem}

What these workshops actually seek to convey, though, is:

\begin{compactitem}
\item
  how to automate repetitive tasks,
\item
  how to track and share work,
\item
  how to grow a program in a modular, testable, reusable way, and
\item
  the difference is between structured and unstructured data.
\end{compactitem}

Software Carpentry's curriculum and teaching practices have been
refined via iterative design, and are informed by current research on
teaching and learning best practices (see \cite{trainingsite} for
details).  Its instructor-training program introduces participants to
a variety of modern teaching techniques (e.g., peer instruction), to
concepts underlying these techniques (e.g., cognitive load theory),
and to topic-specific work by computing education researchers (see
\cite{guzdial2010}, \cite{hazzan2011}, and the first third of
\cite{sorva2012} for overviews.)  One example of how they translate
theory into practice is their insistence on live coding during
teaching as a way of demonstrating and transferring authentic practice
to learners.

Software Carpentry has been assessing learning outcomes and retention
since the beginning of its Sloan Foundation funding in January 2012.
The first round of assessment included both qualitative and
quantitative assessment by Dr.\ Jorge Aranda (then at the University
of Victoria) and Prof.\ Julie Libarkin (Michigan State University).

Dr.\ Aranda surveyed and interviewed participants, observed a
workshop, and analyzed screencasts of participants working through a
programming assignment. He found significant increases in
participants' understanding and use of shell commands, version control
tools, Python, and testing techniques. Perhaps more importantly,
participants reported better proficiency with software tools; greater
concern for issues of provenance and code quality; better strategies
to approach software development; and new research questions that have
become accessible thanks to an increase in participants' software
development skills.

Prof.\ Libarkin performed a more detailed assessment of participants
in a workshop held there, which was attended remotely by students from
the University of Texas at Austin. 85\% of participants reported that
they learned what they hoped to learn, 81\% changed their
computational understanding, and 96\% said they would recommend the
workshop to others.

An attempt to scale this up in 2013 was set back by personnel changes,
but systematic follow-ups with past participants in workshops have now
been resumed, and we expect to be able to present preliminary results
by mid-2014.

\subsection{The Hacker Within}\vspace{\hup}

The Hacker Within (THW)\cite{huff2011}, was founded by graduate students in 
nuclear engineering at the University of Wisconsin to provide a forum for 
sharing scientific computing skills and best practices with their peers. As it 
matured as a student organization, it attracted students from many scientific 
disciplines and academic levels. THW conducted bi-weekly seminars and developed 
a series of short courses addressing programming languages C++, Python, and 
Fortran, and best practices such as version control and test-driven code 
development, as well as basic skills such as UNIX mobility. This curriculum was 
delivered primarily as interactive short workshops on campuses and during 
scientific conferences. Many previous founders of the Hacker Within have since 
become instructors with Software Carpentry and a new generation of THW graduate 
students has begun to emerge in their place. In 2013, new branches of THW were 
begun at the University of Melbourne and the University of California - 
Berkeley.  

\section{Broader Impact}

We believe this work will have significant impact in several related
areas.

\begin{enumerate}

\item
  \emph{Enhance economic competitiveness.} Computing is no longer
  optional in any part of science: even scientists who don't think of
  themselves as doing computational work rely on computers to prepare
  papers, store data, and collaborate with colleagues.  The better
  their computing skills are, the better able they will be to
  contribute to the research that underpins the nation's economic
  competitiveness.

\item
  \emph{Improving STEM education for everyone, not just participants.}
  By creating and validating high-quality open access teaching
  materials, and the methods used to deliver them, this project will
  enable improvement in STEM education for everyone, everywhere, not
  just for participating students and participating institutions.

\item
  \emph{Improving STEM education tomorrow, not just today.}  As noted
  in the introduction of this proposal, most of today's efforts to
  transfer computational skills to STEM researchers and connect them
  with 21st Century innovations in how science is done are flying
  blind: there is effectively no feedback from long-term impact to
  instructional action.  By creating and validating such a feedback
  loop---i.e., by showing scientists how to apply science to their
  teaching---this project will demonstrate how STEM education can be
  continuously improved.

\item
  \emph{Improve participation in STEM by women and underrepresented
    minorities.} The disproportionately low participation of women and
  some minority groups in STEM is well documented, as is the fact that
  computing is one of the least diverse fields within STEM.  This
  second fact creates a vicious circle: people with weaker computing
  skills may be less competitive in research than their peers, which
  reduces their participation in activities viewed as non-core, which
  in turn results in them having weaker skills.  This project will
  strive to break this cycle by giving at-risk students an opportunity
  to ``level up'' in a supportive environment, and by connecting them
  with mentors who can serve as role models.

\end{enumerate}

\section{Career Management Plan}

The graduate students who are serving as mentors for the
undergraduates at the different universities will each be paired with
a local faculty mentor. The faculty mentor will meet regularly with
the graduate student to discuss and problem solve any issues that the
graduate student or undergraduates are having, and to provide active
mentoring on how to train students in computational approaches.

In addition to engaging with the graduate students on their mentoring
of the undergraduates, the faculty mentors will also serve as a
mentors for computational aspects of the graduate students' research
and careers. In many areas of science, computationally-minded students
are located in labs where the PIs do not have strong computational
backgrounds. This means that they do not have a mentor to teach them
about good computational practice in research. In addition, they do
not have someone to discuss computational careers with, thus limiting
their exposure to career paths outside of academia. Because the
faculty mentors will have strong computational backgrounds themselves,
they can fill this void for computationally-minded students.



%%%  END PROJECT NARRATIVE   (15 page limit)
% --------------- Excluded from page limitations
\appendix

\newpage
\pagenumbering{arabic}
\renewcommand{\thepage} {\footnotesize References\,---\,\arabic{page}}

% ------------------------------------------------------------------- References
\small

\bibliographystyle{plain}
\bibliography{./proposal}

% ------------------------------------------------------------------- Data management plan (2 pages max)
\newpage
\pagenumbering{arabic}
\renewcommand{\thepage} {\footnotesize Data management\,---\,\arabic{page}}

\section*{Data Management Plan}

This project will collect human subjects data consisting of background
demographics (including age, gender, ethnicity, field of study, and
prior computing experience), career information (including
undergraduate and/or graduate education, publication histories, and
industrial experience), and degree and kind of participation in open
and web-enabled science projects and communities.  This data will be
used to determine what impact the training provided by this project
has on participants' careers and productivity.

The demographic data will be collected from questionnaires and
interviews administered by the PI and student/postdoctoral researchers
associated with this project, and will be entered into an electronic
database.  Since this data will be from human subjects, approval for
human subjects research will be obtained through the University of
California - Berkeley Review Board.  To ensure confidentiality, each
subject will be assigned an arbitrary code, which will be associated
with all data relating to that subject.  One file that contains the
correspondence between subject names and codes will be kept in an
encrypted password-controlled file accessible only to the PI and
assessment researcher.  The de-identified electronic data will be
preserved on DVDs and external hard drives.

If requested, access to the de-identified data will be provided by
contacting the PI. Data will in principle be available for access and
sharing as soon as is reasonably possible, normally not longer than
one year after publication of the data. The data will be preserved for
at least three years beyond the award period, as required by NSF
guidelines.

We do not anticipate that significant intellectual property issues
involved with this data will arise. However, in the event that
discoveries or inventions are made in direct connection with these
data, access to the data will be granted upon request once appropriate
invention disclosures and/or provisional patent filings are made.  The
data acquired and preserved in the context of this proposal will be
further governed by the University of California - Berkeley's policies
pertaining to intellectual property, record retention, and data
management.

\subsection*{Educational materials}\vspace{\hup}

\comment{LB: Add here a paragraph about the learning objects that will
  be created and how these will be shared openly on the web, how they
  will be version-controlled, etc. Mention the provisions for re-use,
  license, etc.}

\end{document}
