% FIXME: two cultures (Segal)
\documentclass[11pt]{article}

\title{
  Assessing the Impact of Intensive Software Skills Training
  on Students' Scientific Careers
}

\author{
  Rachel Slaybaugh, Kaitlin Thaney, Lorena Barba, C. Titus Brown,\\   
  and Paul Wilson\\ 
  with Ethan White, Tracy Teal, and Greg Wilson
}

\newcommand{\fixme}[2]{FIXME (#1): {#2}}

\begin{document}
\maketitle
\pagebreak
% This is submitted as a separate section in Fast Lane
%\section{Summary}
%
%FIXME: one-page summary
%
%\pagebreak

\section{The Problem}

Scientists and engineers invented electronic computers to accelerate
their work, but two generations later, many researchers in science,
technology, engineering, and mathematics (STEM) are still not
\emph{computationally competent}: they do repetitive tasks manually
instead of automating them, develop software using a methodology best
summarized as ``copy, paste, tweak, and pray'', and fail to track
their work in any systematic, reproducible way.

And while the World-Wide Web was created by a scientist to help his
peers share information, many still use it primarily as a way to find
and download PDFs.  Researchers may understand that open data can fuel
new insights, but often lack the skills needed to create and provide a
reusable data set.  Equally, any discussion of changing scientific
publishing, make research reproducible, or using the web to support
``science as a service'' must eventually address the lack of
pre-requisite skills in the general STEM research community.

Since the mid-1980s (at least), proponents of computational science
have taken an ``if we build it, they will come'' approach to this
problem.  It is clear now, though, that learning-by-osmosis has not
worked, and is unlikely to in the future for several reasons:

\begin{enumerate}

\item
  \emph{The curriculum is full.}  Undergraduate STEM programs already
  struggle to cover material regarded as core to their field.  While
  many scientists would agree that more material on programming,
  reproducible research, or web-enabled science would be useful, there
  is no consensus on what to take out to make room.

\item
  \emph{The blind leading the blind.}  Many faculty lack computational
  skills themselves, and hence are unable to pass them on.

\item
  Scientists and software developers have different cultures,
  different priorities, and different approaches to problem solving,
  which often impedes collaboration and knowledge transfer
  \cite{segal2005a}.

\item
  \emph{Difficulty of assessing impact.} It is easy to say, ``This
  discovery could not have been made without use of that
  supercomputer.''  It is much harder to attribute specific advances
  in science to prior training in general computing skills.

\end{enumerate}

The final, and possibly largest, issue is that \emph{the rewards are
  unknown}.  Open, web-based science is still in its infancy, so there
is no general understanding of what people might need to know in order
to incorporate it into their research careers.  Since it is hard to
measure something if you don't know what to look for, or if it is so
young that there hasn't actually \emph{been} long-term impact, little
systematic study has been done to date of whether early training in
the skills needed to practice open, web-enabled science actually has
an impact, and if so, how and how much.  Without such feedback, there
is no systematic way to improve the training programs that currently
exist.

\section{Related Work}

\subsection{Research}

Studies of how scientists use computers and the web have found that
most scientists learn what they know about developing software and
using computers and the web in their research through osmosis and word
of mouth \cite{hannay2009,prabhu2011}. In our experience, most
training meant to address this issue:

\begin{itemize}

\item
  does not target scientists' specific needs (e.g., is a general
  ``Introduction to Computing'' class shared with students majoring in
  other areas);

\item
  only covers the mechanics of programming in a particular language
  rather than giving a complete picture including data management,
  web-enabled publishing, the ``defense in depth'' approach to
  correctness discussed in \cite{dubois2005}, or the other
  foundational skills laid out in \cite{wilson2013}; and/or

\item
  jumps to advanced topics such as parallel computing before
  scientists have mastered the foundations.  (Most research on
  scientific computing, such as \cite{hochstein2005}, does the same.)

\end{itemize}

\subsection{Software Carpentry}

Software Carpentry \cite{swcsite,wilson2012} is the largest effort to
date to address these issues. Originally created as a training program
at Los Alamos National Laboratory in the late 1990s, it is now part of
the Mozilla Science Lab's efforts to help scientists take advantage of
ways in which the web can change the practice of science today, and
invent new ways tomorrow.  Over 100 certified volunteer instructors
delivered two-day intensive workshops to more than 4200 people in 2013
alone.  Their subject matter usually includes:

\begin{itemize}
\item
  the Unix shell,
\item
  version control with Git,
\item
  programming in Python or R, and
\item
  using SQL.
\end{itemize}

What these workshops actually seek to convey, though, is:

\begin{itemize}
\item
  how to automate repetitive tasks,
\item
  how to track and share work,
\item
  how to grow a program in a modular, testable, reusable way, and
\item
  the difference is between structured and unstructured data.
\end{itemize}

Software Carpentry's curriculum and teaching practices have been
refined via iterative design, and are informed by current research on
teaching and learning best practices.  Its instructor training program
introduces participants to a variety of modern teaching techniques
(e.g., peer instruction), to concepts underlying these techniques
(e.g., cognitive load theory), and to topic-specific work by computing
education researchers (see \cite{guzdial2010}, \cite{hazzan2011}, and
the first third of \cite{sorva2012} for overviews.)  One example of
how we translate theory into practice is our insistence on live coding
during teaching as a way of demonstrating and transferring authentic
practice to learners.

Software Carpentry has been assessing learning outcomes and retention
since the beginning of its Sloan Foundation-funded in January 2012.
The first round of assessment included both qualitative and
quantitative assessment by Dr.\ Jorge Aranda (then at the University
of Victoria) and Prof.\ Julie Libarkin (Michigan State University).
An attempt to scale this up in 2013 was set back by personnel changes,
but systematic follow-ups with past participants in workshops have now
been resumed, and we expect to be able to present preliminary results
by mid-2014.

\subsection{The Hacker Within}

\fixme{PW/KH}{Please fill in.}

\section{Proposed Work}

This proposal builds on our success to date in enhancing the skills of
graduate students, post-docs, and faculty.  We designed it to:

\begin{enumerate}

\item
  conduct formative evaluation of the impact of software skills
  training for undergraduates likely to continue in research careers
  as they progress through the early stages of those careers;

\item
  conduct summative evaluation of the training's overall impact on a
  multi-year timescale in order to improve the content and
  presentation of the training; and

\item
  disseminate the resulting curriculum as widely as possible.

\end{enumerate}

More specifically, we will run two-day software skills workshops each
year for four years for undergraduate students taking part in that
year in the NSF's Research Experience for Undergraduates (REU)
program, at or near the start of those students' time in the lab.  We
believe this training will help them be more productive during their
REU (graduate-level participants in our existing workshops typically
report that what we teach saves them a day a week), and also prepare
them to work in a world where all aspects of science are increasingly
dependent on computing.  More importantly, these undergraduates will
serve as the treatment population for a five-year study of the impact
of this training on their careers in general, and their involvement
with open and web-enabled science in particular.

The sections below detail the specific activities we will undertake.

\subsection{Workshops}

We will run two-day workshops at a steadily increasing number of sites
each year for four years, timed to coincide with the start of the
summer REU influx.  Each workshop will be offered to a minimum of 40
learners per site (giving us a target study population of 1440
students by year 4).  All of the
workshop instructors will have been trained and certified by Software
Carpentry, and will have had prior experience teaching this material.
The content will be tailored to meet local
needs, but will be based on what is being used at that time by
Software Carpentry and affiliated educational efforts.  By design, it is straightforward to adapt workshop materials and contribute changes back to the Software Carpentry organization.  These features enhance the portability and flexibility of the workshops, and increase the likelihood of wide dissemination beyond this project.  

The home sites for investigators named in this proposal (Michigan
State University, Utah State University, George Washington University,
University of New Mexico, University of California -- Berkeley, and
University of Wisconsin -- Madison) will run workshops in each of
those years.  Two other NSF REU sites will be added each year,
increasing the total to 12 sites by year 4, which increases the size of our study population.

In order to increase the diversity of the study population, we will
additionally run at least one workshop in each of years 1-4 that is
specifically aimed at female students.  Software Carpentry's first
such workshop, held in Boston in June 2013, attracted 120
participants; its second is scheduled for Lawrence Berkeley National
Laboratory in March 2014, and at least two more will be held by the
time work on this project commences (one in the United States and one
in Europe).  This work will build on that experience, and draw on the
pool of instructors who have gained mentoring experience through those
specific workshops.

Finally, we will organize an equal number of workshops specifically
aimed at students from minority groups that are underrepresented in
STEM.  We are already in contact with the Computing Alliance for
Hispanic-Serving Institutions (CAHSI), and with the Association of
Public and Land-grant Universities' program for historically black
colleges and universities (HBCUs); Software Carpentry is running its
first workshop at an HBCU (Spelman) in early 2014, and we expect to
have expanded these efforts by the start of this project.

\subsection{Assessment}

We will employ one full-time researcher for five years to monitor and
compare undergraduate participants in these workshops, participants in
a subset of our regular (graduate-level) workshops, and
non-participants (as a control population).  Assessment will focus
particularly, but not exclusively, on the following questions:

\begin{enumerate}

\item
  Are students who receive this training more likely to continue to
  graduate school than their peers?

\item
  Are students who receive this training more likely than their peers
  to incorporate open science and/or web-enabled science tools and
  practices into their work?

\item
  Are students who receive this training more likely than their peers
  to choose computationally-oriented research topics and/or careers?
  Are those who do not choose computationall-oriented paths
  nevertheless more likely to incorporate the tools and practices
  mentioned above into their work?

\item
  Are students who receive this training more likely than their peers
  to develop new tools and practices, and/or become involved in
  outreach and education activities (i.e., are they more likely to
  become creators and leaders)?

\item
  Do outcomes differ between women and underrepresented minorities on
  one hand and non-underrepresented minorities and men on the other?
  If so, in what ways, and what steps are effective in correcting for
  these differences?

\item
  In what ways does this training change students' outlook on the
  practice of science itself?

\end{enumerate}

This researcher will also explore ways in which our engagement with
students changes the outlook and work practices of their peers and
faculty supervisors (i.e., whether there is knowledge transfer
sideways and upward), and at the effectiveness of the community
building and dissemination activities detailed in the next sections.

As with our work to date, assessment will use both qualitative and
quantitative techniques.  On the qualitative side, we will conduct a
series of interviews over the five-year period of the study to see how
attitudes, aspirations, and activities change.  Quantitatively, we
will measure uptake of key tools such as version control as a proxy
for adoption of related practices, as well as exploring more
traditional measures of research success such as progression to
graduate school and publication/citation rates.  

Our findings will be shared with other researchers through publication
in peer-reviewed journals such as \emph{Science Education} and
\emph{Physics Education}, and through presentations at conferences
such as SIGCSE (the ACM's Special Interest Group on Computer Science
Education) and smaller, domain-specific venues including the
Association for Biology Laboratory Education (ABLE), the Society of
Industrial and Applied Mathematics (SIAM)'s education session at the
Joint Mathematics Meeting, and the tracks hosted by the Education
Training and Workforce Development Division at the American Nuclear
Society (ANS)'s annual meeting.  We will also continue to share our
ideas in publications such as \emph{American Scientist}, which reach a
broader cross-section of interested parties.

\subsection{Community Building}

We will employ one graduate student part-time at each participating
site each year to provide technical support to workshop participants,
and to act as an anchor for a Hacker Within-style grassroots group at
that site.  These community liaisons will not be study subjects, but
will help us stay in touch with students who are (a key requirement
for any longitudinal study).

Separately, the Mozilla Science Lab will focus part of its ongoing
community engagement efforts on the students who have taken part in
our workshops during both the remainder of their undergraduate careers
and afterward in order to ensure that they become part of the broader
open science community.  This may include helping the students
organize and run workshops of their own in subsequent years,
connecting them with other open science projects, introducing them to
potential graduate supervisors who understand and value their new
skills and outlook, etc.

As a subordinate part of their work, the researcher employed by this
project will assess the effectiveness of these local organizers.  In
particular, they will explore whether seeding activity in this way
leads to the formation of self-sustaining grassroots groups, and if
so, what activities those groups develop on their own, how (and how
effectively) they share discoveries with each other, the extent to
which alumni of this program stay engaged with these groups, and
whether the presence of these groups has a demonstrable impact on
students' career paths in general, and their engagement with open and
web-enabled science in particular.

\subsection{Curriculum Development and Dissemination}

We will employ one instructional designer part-time during each of the
study's first four years to create new material, and to improve
existing material based on feedback from workshop participants and the
assessment program.  Here, ``creating material'' may include both
designing and implementing new domain-specific learning modules, and
translating existing materials into new forms, such as video
recordings of lectures or auto-graded quizzes for self-paced
instruction.  This work will be done in consultation with educators at
participating institutions in order to encourage incorporation of
those materials into existing curricula.

All of the materials produced by and for this project will be made
freely available under the Creative Commons -- Attribution (CC-BY)
license.  The instructional designer will work with the Mozilla
Science Lab and affiliated groups to share these materials, and the
results of our studies of the program's impact, through science
education journals, conferences, and other channels. Specifically, we plan to publish articles in the following journals 
\begin{itemize}
\item Journal of Computers in Mathematics and Science Teaching, and
%
\item Physics Education,
\end{itemize}
%
and present papers at the following conferences
\begin{itemize}
\item  the Special Interest Group on Computer Science Education (SIGCSE) of the Association for Computing Machinery, Inc. (ACM),
%
\item the International Conference on Physics Education (ICPE),
%
\item the Association for Biology Laboratory Education (ABLE),
%
\item the Society for Industrial and Applied Mathematics (SIAM), and
%
\item the American Nuclear Society (ANS).
\end{itemize}

The dissemination of this project's curriculum has strong potential to be high. Workshop materials are all open access and flexible, thus they can be readily adopted by others. Adapting workshop materials is low cost and does not require special equipment. Workshop materials are structured such that they can scale to the size and application of interest to a particular group. The Software Carpentry infrastructure provides support in the form of materials and people. Anyone using workshop materials can directly contribute changes and feedback, which both increases buy-in and improves the materials organically. And finally, local chapters of The Hacker Within create a natural ecosystem of support for workshop participants, their peers, and faculty. 

As a subordinate part of their work, the researcher employed by this
project will assess the extent to which curriculum developed during
this program is taken up by other educators (particularly those who
think of themselves as scientists first and computationalists second),
and their perception of its utility.  Mid-point results of this
evaluation will be shared with the instructional designer in order to
allow evidence-based improvement of the materials.

\section{Broader Impact}

We believe this work will have significant impact in several related
areas.

\begin{enumerate}

\item
  \emph{Enhance economic competitiveness.} Computing is no longer
  optional in any part of science: even scientists who don't think of
  themselves as doing computational work rely on computers to prepare
  papers, store data, and collaborate with colleagues.  The better
  their computing skills are, the better able they will be to
  contribute to the research that underpins the nation's economic
  competitiveness.

\item
  \emph{Improving STEM education for everyone, not just participants.}
  By creating and validating high-quality open access teaching
  materials, and the methods used to deliver them, this project will
  enable improvement in STEM education for everyone, everywhere, not
  just for participating students and participating institutions.

\item
  \emph{Improving STEM education tomorrow, not just today.}  As noted
  in the introduction of this proposal, most of today's efforts to
  transfer computational skills to STEM researchers and connect them
  with 21st Century innovations in how science is done are flying
  blind: there is effectively no feedback from long-term impact to
  instructional action.  By creating and validating such a feedback
  loop---i.e., by showing scientists how to apply science to their
  teaching---this project will demonstrate how STEM education can be
  continuously improved.

\item
  \emph{Improve participation in STEM by women and underrepresented
    minorities.} The disproportionately low participation of women and
  some minority groups in STEM is well documented, as is the fact that
  computing is one of the least diverse fields within STEM.  This
  second fact creates a vicious circle: people with weaker computing
  skills may be less competitive in research than their peers, which
  reduces their participation in activities viewed as non-core, which
  in turn results in them having weaker skills.  This project will
  strive to break this cycle by giving at-risk students an opportunity
  to ``level up'' in a supportive environment, and by connecting them
  with mentors who can serve as role models.

\end{enumerate}

\section{Career Management Plan}

The graduate students who are serving as mentors for the
undergraduates at the different universities will each be paired with
a local faculty mentor. The faculty mentor will meet regularly with
the graduate student to discuss and problem solve any issues that the
graduate student or undergraduates are having, and to provide active
mentoring on how to train students in computational approaches.

In addition to engaging with the graduate students on their mentoring
of the undergraduates, the faculty mentors will also serve as a
mentors for computational aspects of the graduate students' research
and careers. In many areas of science, computationally-minded students
are located in labs where the PIs do not have strong computational
backgrounds. This means that they do not have a mentor to teach them
about good computational practice in research. In addition, they do
not have someone to discuss computational careers with, thus limiting
their exposure to career paths outside of academia. Because the
faculty mentors will have strong computational backgrounds themselves,
they can fill this void for computationally-minded students.

\section{Data Management Plan}

\fixme{TT}{Talk about curating our data.}

\bibliographystyle{plain}
\bibliography{proposal}

\end{document}
