\documentclass{proposalnsf}

%--------------------------------------------------------------------  PROCESS WITH XeLaTeX
\usepackage{fontspec}% provides font selecting commands 
\usepackage{xunicode}% provides unicode character macros 
\usepackage{xltxtra} % provides some fixes/extras 
\setromanfont[Mapping=tex-text,
                 SmallCapsFont={Palatino},
                 SmallCapsFeatures={Scale=0.85}]{Palatino}
\setsansfont[Scale=0.85]{Trebuchet MS} 
\setmonofont[Scale=0.85]{Monaco}

\renewcommand{\captionlabelfont}{\bf\sffamily}
\usepackage[hang,flushmargin]{footmisc} 
% 'hang' flushes the footnote marker to the left,  'flushmargin'  flushes the text as well.

% Define the color to use in links:
\definecolor{linkcol}{rgb}{0.459,0.071,0.294}
\definecolor{sectcol}{rgb}{0.63,0.16,0.16} % {0,0,0}
\definecolor{propcol}{rgb}{0.75,0.0,0.04}

\definecolor{blue}{rgb}{0,0,0}
\definecolor{green}{rgb}{0.5,0.5,0.5}
\definecolor{gray}{rgb}{0.25,0.25,0.25}
\definecolor{ngreen}{rgb}{0.7,0.7,0.7} % a darker shade of green

\usepackage[
    xetex,
    pdftitle={NSF proposal},
    pdfauthor={Gregory V.\ Wilson},
    pdfpagemode={UseOutlines},
    pdfpagelayout={TwoColumnRight},
    bookmarks, bookmarksopen,bookmarksnumbered={True},
    pdfstartview={FitH},
    colorlinks, linkcolor={sectcol},citecolor={sectcol},urlcolor={sectcol}
    ]{hyperref}

%% Define a new style for the url package that will use a smaller font.
\makeatletter
\def\url@leostyle{%
  \@ifundefined{selectfont}{\def\UrlFont{\sf}}{\def\UrlFont{\small\ttfamily}}}
\makeatother
%% Now actually use the newly defined style.
\urlstyle{leo}


% this handles hanging indents for publications
\def\rrr#1\\{\par
\medskip\hbox{\vbox{\parindent=2em\hsize=6.12in
\hangindent=4em\hangafter=1#1}}}


\addto\captionsamerican{%
  \renewcommand{\refname}%
    {References Cited}%
} % solution found here: http://www.tex.ac.uk/cgi-bin/texfaq2html?label=latexwords

\def\baselinestretch{1}
\setlength{\parindent}{0mm} \setlength{\parskip}{0.8em}

\newlength{\up}
\setlength{\up}{-4mm}

\newlength{\hup}
\setlength{\hup}{-2mm}

\sectionfont{\large\bfseries\color{sectcol}\vspace{-2mm}}
\subsectionfont{\normalsize\it\bfseries\vspace{-4mm}}
\subsubsectionfont{\normalsize\mdseries\itshape\vspace{-4mm}} %\itshape
\paragraphfont{\bfseries}
% ---------------------------------------------------------------------
\begin{document}

% ------------------------------------------------------------------- Biosketch
%\newpage
\pagenumbering{arabic}
\renewcommand{\thepage} {\footnotesize Bio.\,---\,\arabic{page}}
\section*{Dr.\ Gregory V.\ Wilson}

\small
\textbf{Education and training:} 

\begin{tabular}{llcc}
Institution & Major & Degree & Year \\ \hline
Queen's University & Mathematics and Engineering & BSc & 1984 \\
University of Edinburgh & Artificial Intelligence & MSc & 1986 \\
University of Edinburgh & Computer Science & PhD & 1993 \\
\end{tabular}

\textbf{Research and professional experience:} 

\begin{tabular}{ll}
2012 -- present &  Project lead, Software Carpentry, Mozilla Foundation \\
2011         &  Software engineer, Side Effects Software Inc. \\
2010 -- 2011 &  Independent training consultant, Software Carpentry \\
2006 -- 2010 &  Assistant Professor, Dept.\ of Computer Science, University of Toronto \\
2004 -- 2006 &  Independent contractor \\
2000 -- 2004 &  Software engineer, Nevex/Baltimore Technologies/Hewlett-Packard \\
1998 -- 2000 &  Independent contractor \\
1996 -- 1998 &  Software engineer, Visible Decisions Inc. \\
1995 -- 1996 &  Scientist, Centre for Advanced Studies, IBM Toronto \\
1992 -- 1995 &  Post-doctoral researcher at various universities \\
1986 -- 1992 &  Software engineer, Edinburgh Parallel Computing Centre \\
1985         &  Software engineer, Bell-Northern Research \\
1984 -- 1985 &  Software engineer, Miller Communications Ltd. \\
\end{tabular}

\textbf{Honors and awards:} 

\begin{tabular}{ll}
2010 & ComputerWorld Canada's IT Educator of the Year \\
2008 & Jolt Award for Best General Book (with co-editor Andy Oram) \\
2004 & University of Toronto Computer Science Student Union Teaching Award \\
1986 & Howe Prize, University of Edinburgh \\
1985--86 & Commonwealth Scholarship \\
1984 & University Medal, Queen's University (top student in graduating class) \\
1984 & A.B.\ Lillie Prize (top student in mathematics) \\
1982--84 & Dean's Scholar, Faculty of Applied Science, Queen's University \\
\end{tabular}

\textbf{Selected Publications:} % -------------------------------------------- 
%
\vspace{\up}
\begin{list}{$\ast$}{\setlength{\leftmargin}{1em}}

\item
Amy Brown and Greg Wilson (eds.).
{\em The Architecture of Open Source Applications: Elegance, Evolution, and a Few Fearless Hacks}.
Lulu, 2011.

\item
Andy Oram and Greg Wilson (eds.).
{\em Making Software: What Really Works, and Why We Believe It}.
O'Reilly, 2010.

\item
Jo~Erskine Hannay, Hans~Petter Langtangen, Carolyn MacLeod, Dietmar Pfahl, Janice Singer, and Greg Wilson.
How do scientists develop and use scientific software?
In {\em Proceedings of the Second International Workshop on Software Engineering for Computational Science and Engineering (SE-CSE 2009)}. IEEE, 2009.

\item
Greg Wilson.
How do scientists really use computers?
{\em American Scientist}, September/October 2009.

\item
Jordi Cabot and Greg Wilson.
Tools for teams: A survey of web-based software project portals.
{\em Doctor Dobb's Journal}, October 2009.

\item
Jennifer Campbell, Paul Gries, Jason Montojo, and Greg Wilson.
{\em Practical Programming}.
Pragmatic Bookshelf, 2009.

\item
David Matthews, Greg Wilson, and Steve Easterbrook.
Configuration management for large-scale scientific computing at the UK Met Office.
{\em Computing in Science and Engineering}, November/December 2008.

\item
Greg Wilson.
Those who will not learn from history...
{\em Computing in Science and Engineering}, 10(3), May 2008.

\item
Andy Oram and Greg Wilson (eds.).
{\em Beautiful Code: Leading Programmers Explain How They Think}.
O'Reilly, 2007.

\item
Jorge Aranda, Steve Easterbrook, and Greg Wilson.
Requirements in the wild: How small companies do it.
In {\em Proceedings of the 15th International Conference on Requirements Engineering}, October 2007.

\item
Greg Wilson.
Where's the real bottleneck in scientific computing?
{\em American Scientist}, January/February 2006.

\item
D.~Winter, B.~Vinegar, H.~Nahal, R.~Ammar, G.~V. Wilson, and N.~J. Provart.
An 'electronic fluorescent pictograph' browser for exploring and analyzing large-scale biological data sets.
{\em PLoS ONE}, 2(8), 2007.

\item
Greg Wilson.
Open-source offers solutions for science software education.
{\em Nature}, 436:600, July 2005.

\item
Greg Wilson.
{\em Data Crunching: Solve Everyday Problems Using Java, Python, and More}.
Pragmatic Bookshelf, 2005.

\item
Karen~L. Reid and Gregory~V. Wilson.
Learning by doing: introducing version control as a way to manage student assignments.
In {\em Proceedings of the 36th SIGCSE Technical Symposium on Computer Science Education}, pages 272--276. ACM, 2005.

\item
Gregory~V. Wilson and Paul Lu (eds.).
{\em Parallel Programming Using C++}.
MIT Press, 1996.

\item
Eshrat Arjomandi, William~G. O'Farrell, and Gregory~V. Wilson.
Smart messages: An object-oriented communication mechanism for parallel systems.
{\em Computing Systems}, 9(4):313--329, 1996.

\item
Gregory~V. Wilson.
What should computer scientists teach to physical scientists and engineers?
{\em IEEE Computational Science and Engineering}, Summer and Fall 1996.

\item
Gregory~V. Wilson.
{\em Practical Parallel Programming}.
MIT Press, 1995.

\item
Arthur Trew and Greg Wilson (eds.).
{\em Past, Present, Parallel: A Survey of Available Parallel Computing Systems}.
Springer-Verlag, 1991.

\end{list}

\textbf{Synergistic activities:} % --------------------------------------------  

Dr.\ Wilson co-founded Software Carpentry with Brent Gorda in 2004.
Since then,
he has grown it to an international volunteer program
with over 100 instructors in a dozen different countries.
In 2013 alone,
Software Carpentry ran over 90 two-day workshops for almost 4500 scientists.

Dr.\ Wilson has consistently demonstrated a strong commitment to undergraduate education over more than two decades.
He founded the Edinburgh Parallel Computing Centre's Summer Scholarship Programme,
which recruited and trained 60 students between 1988 and 1992.
While at the University of Toronto,
he supervised over 150 undergraduates working alone or in small teams on almost 100 real-world projects,
of which more than half were for clients outside the Computer Science department.
In 2009, he created the UCOSP program,
through which 90 students from more than a dozen Canadian universities worked in distributed teams on open source projects.
He has been a mentor for Google's Summer of Code program since its inception in 2005.

Dr.\ Wilson has been a member of the Python Software Foundation since 2010.

\textbf{Collaborators:} % --------------------------------------------  
\vspace{\up}

\begin{itemize}
\item Marian Petre, Dept.\ of Computer Science, Open University (UK).
\end{itemize}

\textbf{Graduated students and postdocs advised}: Samira Abdi Ashtiani, Aran Donohue, Jeremy Handcock, Carolyn MacLeod, Jason Montojo, Rory Tulk.

\end{document}
