\documentclass[11pt]{article}

\usepackage[top=0.75in, bottom=0.75in, left=1in, right=1in]{geometry}
\pagestyle{empty}
\usepackage{tabu}

\begin{document}
\section{Summary}

Scientists and engineers invented electronic computers to accelerate
their work, but many researchers in science,
technology, engineering, and mathematics (STEM) are still not
\emph{computationally competent}: they do repetitive tasks manually
instead of automating them, develop software using a methodology best
summarized as ``copy, paste, tweak, and pray'', and fail to track
their work in any systematic, reproducible way. Further, scientists and engineers lack the pre-requisite skills to fully participate in open, web-based science. 

Studies of how scientists use computers and the web have found that
most scientists learn what they know about developing software and
using computers and the web in their research through osmosis and word
of mouth. The training meant to address this issue is largely insufficient for a variety of reasons.

Software Carpentry is the largest effort to date to address these
issues. Software Carpentry's curriculum and teaching practices have been
refined via iterative design, and are informed by current research on
teaching and learning best practices. Software Carpentry has been assessing learning outcomes and retention
since the beginning of its Sloan Foundation-funded in January 2012.

To build on this foundation, we will run two-day software skills workshops each
year for four years for undergraduate students taking part in that
year in the NSF's Research Experience for Undergraduates (REU)
program, at or near the start of those students' time in the lab. 
This schedule will reach approximately 1440 students directly.
 We believe this training will help them be more productive during their
REU (graduate-level participants in our existing workshops typically
report that what we teach saves them a day a week), and also prepare
them to work in a world where all aspects of science are increasingly
dependent on computing.  More importantly, these undergraduates will
serve as the treatment population for a five-year study of the impact
of this training on their careers in general, and their involvement
with open and web-enabled science in particular.

In addition, each year we will hold at least one workshop specifically aimed at female students and an equal number of workshops specifically aimed at students from minority groups that are under-represented in
STEM.

We will expand our assessment efforts to compare REU student outcomes
with those of other learners, and to see what impact this training has
on them compared with non-participant peers.  We
will employ one full-time researcher for five years to monitor
undergraduate participants in these workshops, participants in a
subset of our regular (graduate-level) workshops, and non-participants
(as a control population) for comparison purposes.  This researcher
will also investigate whether the outcomes among
underrepresented minorities and women differ from non-underrepresented minorities and men, respectively. Finally, the employed research will explore ways in which our engagement with students changes
the outlook and work practices of their peers and faculty supervisors,
i.e., whether there is knowledge transfer sideways and upward. 

All of the materials produced by and for this project will be made
freely available under the Creative Commons - Attribution (CC-BY)
license.  The instructional designer will work with the Mozilla
Science Lab and affiliated groups to share these materials, and the
results of our studies of the program's impact, through science
education journals, conferences, and other channels.

We believe this work will have significant impact in several related
areas. Improving STEM education for everyone, not just participants. Improving STEM education tomorrow, not just today. Enhance economic competitiveness. Improve participation in STEM by women and under-represented minorities.

\end{document}