\documentclass{proposalnsf}
% This class file has been tweaked to death by LBarba to fit precisely the 
% formatting strictures of NSF, while still being rather pretty.

%--------------------------------------------------------------------  PROCESS WITH XeLaTeX
%\usepackage{fontspec}% provides font selecting commands 
%\usepackage{paralist}       % compactitem environment
%\usepackage{xunicode}% provides unicode character macros 
%\usepackage{xltxtra} % provides some fixes/extras 
%\setromanfont[Mapping=tex-text,
%                 SmallCapsFont={Palatino},
%                 SmallCapsFeatures={Scale=0.85}]{Palatino}
%\setsansfont[Scale=0.85]{Trebuchet MS} 
%\setmonofont[Scale=0.85]{Monaco}

\renewcommand{\captionlabelfont}{\bf\sffamily}
\usepackage[hang,flushmargin]{footmisc} 
% 'hang' flushes the footnote marker to the left,  'flushmargin'  flushes the text as well.

% Define the color to use in links:
\definecolor{linkcol}{rgb}{0.459,0.071,0.294}
\definecolor{sectcol}{rgb}{0.63,0.16,0.16} % {0,0,0}
\definecolor{propcol}{rgb}{0.75,0.0,0.04}

\definecolor{gray}{rgb}{0.25,0.25,0.25}
\definecolor{ngreen}{rgb}{0.7,0.7,0.7} % a darker shade of green

\usepackage[
    %xetex,
    pdftitle={NSF proposal},
    pdfauthor={Rachel Slaybaugh, Kaitlin Thaney, Lorena Barba, C. Titus Brown, 
    Paul Wilson, Ethan White, Tracy Teal, Greg Wilson, and Kathryn Huff},
    pdfpagemode={UseOutlines},
    pdfpagelayout={TwoColumnRight},
    bookmarks, bookmarksopen,bookmarksnumbered={True},
    pdfstartview={FitH},
    colorlinks, linkcolor={sectcol},citecolor={sectcol},urlcolor={sectcol}
    ]{hyperref}

%% Define a new style for the url package that will use a smaller font.
\makeatletter
\def\url@leostyle{%
  \@ifundefined{selectfont}{\def\UrlFont{\sf}}{\def\UrlFont{\small\ttfamily}}}
\makeatother
%% Now actually use the newly defined style.
\urlstyle{leo}


% this handles hanging indents for publications
\def\rrr#1\\{\par
\medskip\hbox{\vbox{\parindent=2em\hsize=6.12in
\hangindent=4em\hangafter=1#1}}}


\addto\captionsamerican{%
  \renewcommand{\refname}%
    {References Cited}%
} % solution found here: http://www.tex.ac.uk/cgi-bin/texfaq2html?label=latexwords

\def\baselinestretch{1}
\setlength{\parindent}{0mm} \setlength{\parskip}{0.8em}

\newlength{\up}
\setlength{\up}{-4mm}

\newlength{\hup}
\setlength{\hup}{-2mm}

\sectionfont{\large\bfseries\color{sectcol}\vspace{-2mm}}
\subsectionfont{\normalsize\it\bfseries\vspace{-4mm}}
\subsubsectionfont{\normalsize\mdseries\itshape\vspace{-4mm}} %\itshape
\paragraphfont{\bfseries}

% ---------------------------------------------------------------------
% DRAFTING COMMENTS:
\newcommand\ignore[1]{} % Styles for author comments:
% Enter a comment like this:   \comment{This is a comment.}
%\ignore{
\newcommand{\important}[1]{\textcolor{red}{ #1 }}
\newcommand{\comment}[1]{\textcolor{blue}{ #1 }}
%}
% Uncomment lines below to change from visible to invisible comments.
%\renewcommand{\important}[1]{}
%\renewcommand{\comment}[1]{}

\begin{document}

\newpage

% ----------------------  PROJECT SUMMARY - 1 page max - write in 3rd person

% Small title with normal-size font:
\begin{center}

\large\textbf{Summary}

\end{center}

\textbf{Overview:}

Computers are now ubiquitous in scientific research, but many
researchers in science, technology, engineering, and mathematics
(STEM) are still not \emph{computationally competent}: they do
repetitive tasks manually instead of automating them, develop software
haphazardly, and fail to track their work in any systematic,
reproducible way.  This lack of foundational skills impedes their
ability to do research, and prevents them from engaging with new
opportunities in open and web-enabled science.

This project will train STEM undergraduates in these skills, and
assess the impact of that training on their productivity and career
paths.  We will do this by running software skills workshops for
undergraduates likely to go on to graduate school, and by tracking
the alumni of these workshops over several subsequent years both to
improve the training itself and to encourage wider adoption of our
model.

More specifically, we will adapt the two-day workshops run by Software
Carpentry for researchers at the graduate level and above to teach
undergraduate students drawn primarily from programs such as the NSF's
Research Experience for Undergraduates (REU) program.  These workshops
will cover fundamental skills that are prerequisites for open and
web-enabled science, including how to automate repetitive tasks, how
to track and share work over the web, how to grow a program in a
modular, testable, reusable way, and how to create, use, and share
structured data.  All materials will be made freely available to other
educators and institutions under an open access license in order to
promote the greatest possible uptake.

Together, these workshops will reach over 2000 students during the
course of the project.  Alumni will serve as the treatment population
for a five-year study by a full-time professional researcher in
educational assessment, who will explore the impact of this training
on their careers in general, and their involvement with open and
web-enabled science in particular.

\textbf{Intellectual merit:}

Our main contribution to knowledge will be an assessment of ways in
which training of this kind can accelerate the careers of participants
and the science they do.  We will discover whether students who
receive this training are more likely than their peers to continue to
graduate school, to incorporate open and web-enabled science tools and
practices into their work, to choose computationally oriented research
topics and careers, and/or to develop new computational tools and
practices.  We will also determine the extent to which training of
this kind changes students' outlook on the practice of science itself,
and whether it can help level the playing field for women and
minorities currently underrepresented in computing and science.

\textbf {Broader impacts:}

In the medium term, this project will have broader impact through the
creation and dissemination of teaching materials and practices that
other educators and institutions can adopt.  All of the materials
produced by and for this project will be made freely available under
the Creative Commons - Attribution (CC-BY) license, while the results
of our studies of the program's effect will be shared with other
educators through science education journals and conferences.

In the long term, the project's greatest impact will be on scientific
competitiveness.  Computing is no longer optional in any part of
science: even scientists who don't think of themselves as doing
computational work rely on computers to prepare papers, store data,
and collaborate with colleagues.  The better their computing skills
are, the better able they will be to conduct world-class research that
aids national economic competitiveness.

\renewcommand{\thepage} {\footnotesize Project Summary}

\end{document}
