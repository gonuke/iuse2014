\documentclass{proposalnsf}
% This class file has been tweaked to death by LBarba to fit precisely the 
% formatting strictures of NSF, while still being rather pretty.

%%--------------------------------------------------------------------  PROCESS WITH XeLaTeX
%\usepackage{fontspec}% provides font selecting commands 
%\usepackage{paralist}       % compactitem environment
%\usepackage{xunicode}% provides unicode character macros 
%\usepackage{xltxtra} % provides some fixes/extras 
%\setromanfont[Mapping=tex-text,
%                 SmallCapsFont={Palatino},
%                 SmallCapsFeatures={Scale=0.85}]{Palatino}
%\setsansfont[Scale=0.85]{Trebuchet MS} 
%\setmonofont[Scale=0.85]{Monaco}

\renewcommand{\captionlabelfont}{\bf\sffamily}
\usepackage[hang,flushmargin]{footmisc} 
% 'hang' flushes the footnote marker to the left,  'flushmargin'  flushes the text as well.

% Define the color to use in links:
\definecolor{linkcol}{rgb}{0.459,0.071,0.294}
\definecolor{sectcol}{rgb}{0.63,0.16,0.16} % {0,0,0}
\definecolor{propcol}{rgb}{0.75,0.0,0.04}

\definecolor{gray}{rgb}{0.25,0.25,0.25}
\definecolor{ngreen}{rgb}{0.7,0.7,0.7} % a darker shade of green

\usepackage[
    %xetex,
    pdftitle={NSF proposal},
    pdfauthor={Rachel Slaybaugh, Kaitlin Thaney, Lorena Barba, C. Titus Brown, Paul Wilson, Ethan White, Tracy Teal, Greg Wilson, and Kathryn Huff},
    pdfpagemode={UseOutlines},
    pdfpagelayout={TwoColumnRight},
    bookmarks, bookmarksopen,bookmarksnumbered={True},
    pdfstartview={FitH},
    colorlinks, linkcolor={sectcol},citecolor={sectcol},urlcolor={sectcol}
    ]{hyperref}

%% Define a new style for the url package that will use a smaller font.
\makeatletter
\def\url@leostyle{%
  \@ifundefined{selectfont}{\def\UrlFont{\sf}}{\def\UrlFont{\small\ttfamily}}}
\makeatother
%% Now actually use the newly defined style.
\urlstyle{leo}


% this handles hanging indents for publications
\def\rrr#1\\{\par
\medskip\hbox{\vbox{\parindent=2em\hsize=6.12in
\hangindent=4em\hangafter=1#1}}}


\addto\captionsamerican{%
  \renewcommand{\refname}%
    {References Cited}%
} % solution found here: http://www.tex.ac.uk/cgi-bin/texfaq2html?label=latexwords

\def\baselinestretch{1}
\setlength{\parindent}{0mm} \setlength{\parskip}{0.8em}

\newlength{\up}
\setlength{\up}{-4mm}

\newlength{\hup}
\setlength{\hup}{-2mm}

\sectionfont{\large\bfseries\color{sectcol}\vspace{-2mm}}
\subsectionfont{\normalsize\it\bfseries\vspace{-4mm}}
\subsubsectionfont{\normalsize\mdseries\itshape\vspace{-4mm}} %\itshape
\paragraphfont{\bfseries}

%\usepackage[top=0.75in, bottom=0.75in, left=1in, right=1in]{geometry}
%\pagestyle{empty}
%\usepackage{tabu}

\begin{document}
\newpage
\pagenumbering{arabic}
\renewcommand{\thepage} {\footnotesize Data Management\,---\,\arabic{page}}

\section*{Data Management Plan}

\subsection*{Human Subject Data}

This project will collect human subjects data consisting of background
demographics (including age, gender, ethnicity, field of study, and
prior computing experience), career information (including
undergraduate and/or graduate education, publication histories, and
industrial experience), and extent and kind of participation in open
and web-enabled science projects and communities.  This data will be
used to determine what impact the training provided by this project
has on participants' careers and productivity.

The demographic data will be collected from questionnaires and
interviews administered by the PI and student/postdoctoral researchers
associated with this project, and will be entered into an electronic
database.  Since this data will be from human subjects, approval for
human subjects research will be obtained through the University of
California - Berkeley Review Board.  To ensure confidentiality, each
subject will be assigned an arbitrary code, which will be associated
with all data relating to that subject.  One file that contains the
correspondence between subject names and codes will be kept in an
encrypted, password-controlled file accessible only to the PI and
assessment researcher.  The de-identified electronic data will be
preserved on DVDs and external hard drives.

If requested, access to the de-identified data will be provided by
contacting the principal investigator. Data will in principle be
available for access and sharing as soon as is reasonably possible,
normally not longer than one year after publication of the data. The
data will be preserved for at least three years beyond the award
period, as required by NSF guidelines.

We do not anticipate that significant intellectual property issues
involved with this data will arise. However, in the event that
discoveries or inventions are made in direct connection with these
data, access to the data will be granted upon request once appropriate
invention disclosures and/or provisional patent filings are made.  The
data acquired and preserved in the context of this proposal will be
further governed by the University of California - Berkeley's policies
pertaining to intellectual property, record retention, and data
management.

\subsection*{Educational Materials}

All of the educational materials produced by and for this project will
be made freely available under the Creative Commons -- Attribution
(CC-BY) license.  Authorship and contribution will be tracked to
ensure that contributors are credited for their work.  All materials
will be archived in publicly accessible version control repositories,
and made available for download from the Software Carpentry web site
and university web sites in widely-used formats, including (but not
limited to) HTML pages, PDF documents, and slide decks compatible with
common presentation tools.  Finally, we shall provide metadata such as
prerequisite descriptions, lesson plans, and exercises in standard
e-learning formats such as SCORM.

\end{document}
