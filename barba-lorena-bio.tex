\documentclass{proposalnsf}

% Document:	NSF proposal
% Author:		Lorena Barba
% biosketch updated Jan. 23, 2014

%--------------------------------------------------------------------  PROCESS WITH XeLaTeX
\usepackage{fontspec}% provides font selecting commands 
\usepackage{mdwlist} % reduces spacing in itemize and enumerate
\usepackage{xunicode}% provides unicode character macros 
\usepackage{xltxtra} % provides some fixes/extras 
\setromanfont[Mapping=tex-text,
                 SmallCapsFont={Palatino},
                 SmallCapsFeatures={Scale=0.85}]{Palatino}
\setsansfont[Scale=0.85]{Trebuchet MS} 
\setmonofont[Scale=0.85]{Monaco}

\renewcommand{\captionlabelfont}{\bf\sffamily}
\usepackage[hang,flushmargin]{footmisc} 
% 'hang' flushes the footnote marker to the left,  'flushmargin'  flushes the text as well.



% Define the color to use in links:
\definecolor{linkcol}{rgb}{0.459,0.071,0.294}
\definecolor{sectcol}{rgb}{0.63,0.16,0.16} % {0,0,0}
\definecolor{propcol}{rgb}{0.75,0.0,0.04}

\definecolor{blue}{rgb}{0,0,0}
\definecolor{green}{rgb}{0.5,0.5,0.5}
\definecolor{gray}{rgb}{0.25,0.25,0.25}
\definecolor{ngreen}{rgb}{0.7,0.7,0.7} % a darker shade of green



\usepackage[
    xetex,
    pdftitle={NSF proposal},
    pdfauthor={Lorena Barba},
    pdfpagemode={UseOutlines},
    pdfpagelayout={TwoColumnRight},
    bookmarks, bookmarksopen,bookmarksnumbered={True},
    pdfstartview={FitH},
    colorlinks, linkcolor={sectcol},citecolor={sectcol},urlcolor={sectcol}
    ]{hyperref}

%% Define a new style for the url package that will use a smaller font.
\makeatletter
\def\url@leostyle{%
  \@ifundefined{selectfont}{\def\UrlFont{\sf}}{\def\UrlFont{\small\ttfamily}}}
\makeatother
%% Now actually use the newly defined style.
\urlstyle{leo}


% this handles hanging indents for publications
\def\rrr#1\\{\par
\medskip\hbox{\vbox{\parindent=2em\hsize=6.12in
\hangindent=4em\hangafter=1#1}}}


\addto\captionsamerican{%
  \renewcommand{\refname}%
    {References Cited}%
} % solution found here: http://www.tex.ac.uk/cgi-bin/texfaq2html?label=latexwords

\newcommand{\nvidia}{\textsc{NVIDIA}\xspace}
\newcommand{\petsc}{\textsc{PETS}c\xspace}
\newcommand{\mpi}{\textsc{MPI}\xspace}
\newcommand{\cfd}{\textsc{CFD}\xspace}
\newcommand{\simd}{\textsc{SIMD}\xspace}
\newcommand{\openmp}{\textsc{O}pen\textsc{MP}\xspace}
\newcommand{\ibm}{\textsc{IBM}}
\newcommand{\pde}{\textsc{PDE}}
\newcommand{\cpu}{\textsc{CPU}}
\newcommand{\gpu}{\textsc{GPU}}
\newcommand{\cuda}{\textsc{CUDA}\xspace}
\newcommand{\mg}{\textsc{MG}\xspace}
\newcommand{\cfl}{\textsc{CFL}\xspace}

\def\baselinestretch{1}
\setlength{\parindent}{0mm} \setlength{\parskip}{0.8em}

\newlength{\up}
\setlength{\up}{-4mm}

\newlength{\hup}
\setlength{\hup}{-2mm}

\sectionfont{\large\bfseries\color{sectcol}\vspace{-2mm}}
\subsectionfont{\normalsize\it\bfseries\vspace{-4mm}}
\subsubsectionfont{\normalsize\mdseries\itshape\vspace{-4mm}} %\itshape
\paragraphfont{\bfseries}
% ---------------------------------------------------------------------
\begin{document}


% ------------------------------------------------------------------- Biosketch
%\newpage
\pagenumbering{arabic}
\renewcommand{\thepage} {\footnotesize Barba bio-sketch\,---\,\arabic{page}}
\section*{Prof.\ Lorena Barba}

\small
\textbf{Professional preparation:} 

\begin{tabular}{llcc}
Institution & Major & Degree & Year \\ \hline
Universidad Tecnica Federico Santa Maria, Chile   &   Mechanical Engineering & BSc & 1989 \\
Universidad Tecnica Federico Santa Maria, Chile   &   Mechanical Engineering & PEng & 1998 \\
California Institute of Technology, Pasadena, CA   &   Aeronautics & MSc & 1999 \\
California Institute of Technology, Pasadena, CA  &   Aeronautics & PhD & 2004 \\
\end{tabular}


\textbf{Appointments:} 

\begin{tabular}{ll}

08/2013-- & Associate Professor, Mechanical and Aerospace Engineering,\\
& The George Washington  University, Washington, DC.\\

09/2008--07/2013 &Assistant Professor, Mechanical Engineering, Boston University, Boston, MA.\\

08/2004--09/2008 & Lecturer in Applied Mathematics, University of Bristol, UK

\end{tabular}


\textbf{Products:} % -------------------------------------------- 

\vspace{\hup}
\begin{list}{$\ast$}{\setlength{\leftmargin}{1em}}

\item ``The Gathering Storm: Flipping the Classroom,'' Lorena A. Barba. SHPE Magazine, Winter 2014, \textbf{1}6(1):28--30 (January 2014), Society of Hispanic Professional Engineers. \\ Online at: \href{http://www.nxtbook.com/nxtbooks/shpe/winter14/#/30}{http://www.nxtbook.com/nxtbooks/shpe/winter14/\#/30}

\item ``CFD Python: 12 Steps to Navier-Stokesequations,'' set of IPython Notebooks for teaching Computational Fluid Dynamics (July 2013), online at \href{https://bitbucket.org/cfdpython/cfd-python-class/}{https://bitbucket.org/cfdpython/cfd-python-class/}

\item ``Everybody's Flippin' ---An update on the flipped classroom,'' Lorena A. Barba (17 February 2013), screencast on figshare, 
\href{http://dx.doi.org/10.6084/m9.figshare.157200}{doi:10.6084/m9.figshare.157200}

\item ``ME 702--Computational Fluid Dynamics,'' Lorena A. Barba (2012), set of 32 lecture screencasts on YouTube, with more than 190,000 collective views (checked January 2014). \\
Online playlist: \href{http://www.youtube.com/playlist?list=PL30F4C5ABCE62CB61}{http://www.youtube.com/playlist?list=PL30F4C5ABCE62CB61}


\end{list}


\vspace{\up}
\begin{list}{$-$}{\setlength{\leftmargin}{1em}}

\item ``Finding the Force---Consistent Particle Seeding for Satellite Aerodynamics,'' J. Brent Parham, L. A. Barba. AIAA Modeling and Simulation Technologies Conference, American Institute of Aeronautics and Astronautics (January 2014), National Harbor, MD, \href{http://dx.doi.org/10.2514/6.2014-1492}{doi:10.2514/6.2014-1492}. Preprint on \href{http://arxiv.org/abs/1312.3691}{arXiv:1312.3691} 

\item ``A biomolecular electrostatics solver using Python, GPUs and boundary elements that can handle solvent-filled cavities and Stern layers,'' Christopher D. Cooper, Jaydeep P. Bardhan, L. A. Barba. \textit{Comput. Phys. Comm.} (2013), \href{http://dx.doi.org/10.1016/j.cpc.2013.10.028}{doi:10.1016/j.cpc.2013.10.028}. Preprint on \href{http://arxiv.org/abs/1309.4018}{arXiv:1309.4018}

\item  Rio Yokota, L.~A.~Barba, Tetsu Narumi, Kenji Yasuoka, Petascale turbulence simulation using a highly parallel fast multipole method, \textit{Comput. Phys. Comm.}, \textbf{184}(3):445--455 (March 2013). \\ \href{http://dx.doi.org/10.1016/j.cpc.2012.09.011}{doi:10.1016/j.cpc.2012.09.011}. Preprint on \href{http://arxiv.org/abs/1106.5273}{arXiv:1106.5273}


\end{list}

\newpage

\textbf{Synergistic activities:} % --------------------------------------------  

\vspace{\up}


\begin{list}{ }{\setlength{\leftmargin}{2.5em}}


	\item[]  Dr Barba has consistently advocated and participated in the open science philosophy.  Within her research group, all software developed is open-source and publicly available, at the different stages of development.  Preprints of all manuscripts are uploaded to the \href{http://arxiv.org/find/cs/1/au:+Barba_L/0/1/0/all/0/1}{arXiv} repository and software projects are distributed via various hosting sites (Google Code, Bitbucket, GitHub).

	Consistent with the open science commitment, Dr Barba also has engaged in the open course-ware movement.  Her latest courses have been made available publicly in the form of (video) screencasts via iTunes U and YouTube.
	
	\item[Jan.'11] \emph{``PASI---Pan-American Advanced Studies Institute.  Scientific Computing in the Americas: The challenge of massive parallelism''} (Valpara\'iso, Chile).  Two-week advanced studies institute, organized entirely by Dr Barba and funded by NSF via award OISE-1036435 (amount \$100,600).  The focus of this PASI is on scientific discovery by means of high-performance computing using the latest many-core computer hardware, in particular graphics processors, or \gpu s.  Bringing this event to Chile aims at building scientific capacity, promoting collaboration, and advanced training of young researchers in HPC. 
	
	\item[Nov.'05--Jul.'09] \emph{``SCAT---Scientific Computing Advanced Training''}: Dr Barba put together an international network of collaboration involving 10 institutions in 6 countries, and prepared a proposal to the European Commission co-operation office for a grant under \emph{Programme ALFA II}, for projects of collaboration of higher-education institutions in Europe and Latin America, for scientific and technical training and knowledge transfer.   The project, one of only six projects funded in 2005 within this program, was awarded a budget of nearly \textbf{\euro 1.4 million} for the whole duration. 
It involved more than 40 professors and researchers, it awarded nearly 30 grants for graduate students and postdocs to travel to a network institution as a visiting scholar (for a collective 235 funded graduate-student months), and sponsored 10 international scientific meetings. Dr Barba hosted 6 of the grantees at Bristol, 4 of which continued studies in PhD programs around the world. 

	\item[\emph{Invited Seminars:}] Dr Barba has visited the following institutions to give seminars: \vspace{\hup}
\item[In 2010:] Columbia University, Brown University, Worcester Polytechnic Institute.\vspace{\hup}
\item[2008--'09:] Harvard University (Initiative in Innovative Computing, IIC), Illinois Institute of Technology, Purdue University, Northwestern University, University of Sussex UK, Instituto Madrileno de Estudios Avanzados IMDEA, Madrid.\vspace{\hup}
\item[2004--'07] University of Delaware, Exeter University UK, Technical University of Eindhoven, Johns Hopkins University,  Bath University UK, ETH Zurich (Institute of Computational Science), University of Southampton UK, University of Leicester UK (Centre for Mathematical Modelling). 
\end{list}


\textbf{Collaborators:} % --------------------------------------------  
\vspace{\up}

\begin{itemize}
\item Jaydeep Bardhan, Department of Electrical and Computer Engineering, Northeastern University

\item Cris Cecka, Institute for Applied Computational Science, Harvard University

\item Tsuyoshi Hamada, Nagasaki University Advanced Computing Center (NACC), Japan
\item Hans Johnston, Mathematics and Statistics Department, University of Massachusetts, Amherst.
\item Matthew G Knepley, Computation Institute, University of Chicago

\item Louis F Rossi, Department of Mathematical Sciences, University of Delaware

\item Jake Socha, Department of Engineering Science and Mechanics, Virginia Tech
\item Pavlos P. Vlachos, Department of Mechanical Engineering, Virginia Tech
\end{itemize}

\textbf{Graduate advisor:} \quad Anthony Leonard, California Institute of Technology.

\textbf{Graduated students and postdocs advised}: Felipe A. Cruz, Simon K. Layton, Rio Yokota.

%%%%%%%%
\end{document}
